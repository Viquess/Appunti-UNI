\documentclass[12pt]{article}

\usepackage{float}
\usepackage{xcolor}
\usepackage{placeins}
\usepackage{listings}
\usepackage{enumitem}
\usepackage{graphicx}
\usepackage{titlesec}
\usepackage{tabularx}
\usepackage{subcaption}
\usepackage[T1]{fontenc}
\usepackage[none]{hyphenat}
\usepackage[most]{tcolorbox}
\usepackage[italian]{babel}
\usepackage[a4paper, margin=75pt]{geometry}
\usepackage[colorlinks=true, linkcolor=black]{hyperref}

\sloppy
\graphicspath{{images/}}
\setlist{itemsep=2pt}
\definecolor{definition}{RGB}{180,230,180}
\definecolor{code_strings}{RGB}{180,30,60}
\definecolor{code_numbers}{RGB}{0,80,160}
\definecolor{code_comments}{RGB}{136,136,136}
\definecolor{code_keywords}{RGB}{128,0,128}
\definecolor{code_brackets}{RGB}{204,102,0}
\definecolor{code_extra}{RGB}{180,80,160}

\tcbset{
  statement/.style={
    colback=white,
    coltitle=black,
    fonttitle=\bfseries
  }
}

\newcounter{definition}[section]
\renewcommand{\thedefinition}{\thesection.\arabic{definition}}
\newtcolorbox{definition}[1][]{
  statement,
  colframe=definition,
  boxed title style={colback=definition},
  before title={\refstepcounter{definition}},
  title={Definizione \thedefinition: #1}
}

\lstset{
  language=Python,
  backgroundcolor=\color[HTML]{F2F2F2},
  commentstyle=\color{code_comments},
  keywordstyle=\color{code_keywords},
  numberstyle=\tiny\color[RGB]{128,128,128},
  stringstyle=\color{code_strings},
  basicstyle=\ttfamily\footnotesize,
  upquote=true,
  breakatwhitespace=false,
  breaklines=true,
  captionpos=b,
  keepspaces=true,
  numbers=left,
  numbersep=5pt,
  showspaces=false,
  showstringspaces=false,
  showtabs=false,
  tabsize=2,
  frame=lines,
  rulecolor=\color[HTML]{F2F2F2}
}

\lstdefinelanguage{HTTP}{
  sensitive=true,
  morecomment=[l]{\#},
  morestring=[b]",
  morekeywords={GET, POST, PUT, DELETE, HEAD, OPTIONS, PATCH, TRACE, CONNECT},
  columns=fullflexible,
  alsoletter={:},
  moredelim=[is][\color{code_extra}\bfseries]{~}{~},
  literate=
    *{HTTP/1.0}{{{\color{code_keywords}HTTP/1.0}}}1
     {HTTP/1.1}{{{\color{code_keywords}HTTP/1.1}}}1
     {HTTP/2}{{{\color{code_keywords}HTTP/2}}}1
     {HTTP/3}{{{\color{code_keywords}HTTP/3}}}1
}

\lstdefinelanguage{JSON}{
    keywords={true,false,null},
    morestring=[b]",
    literate=
     *{0}{{{\color{code_numbers}0}}}{1}
      {1}{{{\color{code_numbers}1}}}{1}
      {2}{{{\color{code_numbers}2}}}{1}
      {3}{{{\color{code_numbers}3}}}{1}
      {4}{{{\color{code_numbers}4}}}{1}
      {5}{{{\color{code_numbers}5}}}{1}
      {6}{{{\color{code_numbers}6}}}{1}
      {7}{{{\color{code_numbers}7}}}{1}
      {8}{{{\color{code_numbers}8}}}{1}
      {9}{{{\color{code_numbers}9}}}{1}
      {\{}{{{\color{code_brackets}{\{}}}}{1}
      {\}}{{{\color{code_brackets}{\}}}}}{1}
      {[}{{{\color{code_brackets}{[}}}}{1}
      {]}{{{\color{code_brackets}{]}}}}{1},
}

\lstdefinelanguage{YAML}{
    keywords={true,false,null},
    morestring=[b]",
    morecomment=[l]{\#},
    moredelim=[is][\color{strings}]{~}{~},
    literate=
     *{0}{{{\color{code_numbers}0}}}{1}
      {1}{{{\color{code_numbers}1}}}{1}
      {2}{{{\color{code_numbers}2}}}{1}
      {3}{{{\color{code_numbers}3}}}{1}
      {4}{{{\color{code_numbers}4}}}{1}
      {5}{{{\color{code_numbers}5}}}{1}
      {6}{{{\color{code_numbers}6}}}{1}
      {7}{{{\color{code_numbers}7}}}{1}
      {8}{{{\color{code_numbers}8}}}{1}
      {9}{{{\color{code_numbers}9}}}{1}
      {\{}{{{\color{code_brackets}{\{}}}}{1}
      {\}}{{{\color{code_brackets}{\}}}}}{1}
      {[}{{{\color{code_brackets}{[}}}}{1}
      {]}{{{\color{code_brackets}{]}}}}{1},
}

\lstnewenvironment{code}[1][]
  {\lstset{#1}}
  {}

\titleformat{\section}
  [block]
  {\raggedleft\LARGE\bfseries}
  {\textcolor{gray}{\scalebox{5}{\thesection}}}
  {0pt}
  {\\[3pt]}

\titlespacing*{\section}
  {0pt}   % rientro sinistro
  {0pt}   % spazio prima della sezione
  {30pt}  % spazio dopo la sezione


\begin{document}
    % --- Copertina ---
    \begin{titlepage}
        \centering

        % --- Logo Sapienza ---
        \includegraphics[width=0.95\textwidth]{logo_sapienza.png}
        
        \vspace*{\stretch{0.2}}
        
        % --- Dati Sapienza ---
        {\LARGE "Sapienza" Università di Roma}\\[3pt]
        {\Large Ingegneria dell'Informazione, Informatica e Statistica}\\[3pt]
        {\large Dipartimento di Informatica}\\[3pt]
        
        % --- Spazio vuoto ---
        \vspace*{\stretch{1}}
        
        % --- Titolo ---
        \hrulefill\\
        \vspace{15pt}
        \textbf{\huge Programmazione WEB}\\
        \vspace{7pt}
        \hrulefill\\
        
        % --- Spazio vuoto ---
        \vspace*{\stretch{2}}
        
        % --- Autore ---
        \textit{\Large Autore}\\[3pt]
        {\Large Vincenzo Bova}\\
        
        % --- Spazio vuoto ---
        \vspace*{\stretch{1}}

        % --- Data ---
        {\large A.A. 2025/2026}\\
    \end{titlepage}

    % --- Indice ---
    \newpage
    \tableofcontents

    % --- Introduzione a Git ---
    \newpage
    \section{Introduzione a Git}
    \subsection{Sistemi di versionamento}
    Durante lo sviluppo di un progetto c'è spesso la necessità di effettuare revisioni, correzioni o modifiche ai file che lo compongono.
    \begin{figure}[H]
        \centering
        \includegraphics[width=0.5\textwidth]{introduzione_a_git/no_git_example.png}
    \end{figure}
    Gestire ciò creando ogni volta nuovi file, tuttavia, comporta evidenti problemi:
    \begin{itemize}
      \item \textbf{Duplicazione del contenuto:} che rende il sistema inefficiente e aumenta la difficoltà nel mantenere integrità;
      \item \textbf{Assenza di Naming Convention:} che rende impossibile risalire ad uno storico delle modifiche;
      \item \textbf{Autori incerti};
      \item ...
    \end{itemize}
    \begin{samepage}
      Per ovviare a ciò sono stati creati i \textbf{sistemi di versionamento} (git, csv, mercurial, svn...), i quali offrono vari benefici:
      \begin{itemize}
        \item \textbf{Gestione delle versioni:} il sistema si occupa automaticamente di etichettare le varie versioni in modo consistente;
        \item \textbf{Tracciamento delle mofiche:} è possibile accedere ad uno storico delle modifiche effettuate;
        \item \textbf{Presenza di metadati:} ogni modifica ha un autore, una data...;
        \item \textbf{Creazione di linee di sviluppo parallele:} è possibile creare una versione parallela del codice per non modificare la versione principale, e poi riunirle integrando i cambiamenti;
        \item \textbf{Sincronizzazione tra computer:} il sistema consente di mantenere il progetto allineato tra più computer.
      \end{itemize}
    \end{samepage}

    \subsection{Git}
    Git è un sistema di versionamento distribuito e veloce, creato nel 2005 e capace di gestire progetti di grandi dimensioni.
    Si basa su un design semplice e utilizza DAG (\textit{Directed Acyclic Graph}) e Merkle trees come strutture dati.
    \begin{definition}[Repository]
      È un insieme di commit, branch e tag.\\
      Per semplicità assumiamo che un progetto equivale ad un repository.
    \end{definition}
    \begin{definition}[Working copy]
      È un insieme di file tracciati nella copia locale.\\
      Un nuovo file non sarà ancora tracciato e bisognerà aggiungerlo.\\
      Quando aggiorniamo i file (\textit{update}), stiamo aggiornando la working copy.
    \end{definition}
    
    \subsubsection{Commit}
    Un commit è un'istantanea del repository in un determinato momento.\\
    Viene identificato dallo \textbf{SHA1} del commit stesso e contiene diversi campi:
    \begin{itemize}
      \item data + autore, data + commiter
      \item commento \textbf{obbligatorio}
      \item 0,1 o più genitori
      \item tree: hash di tutti i file nel commit
    \end{itemize}
    \begin{minipage}{\textwidth}
      In particolare il commit può contenere un sottoinsieme delle modifiche (anche ad un singolo file), le quali devono essere aggiunte alla staging area dei cambiamenti.
      \begin{figure}[H]
        \centering
        \begin{subfigure}{0.49\textwidth}
          \centering
          \includegraphics[height=4cm]{introduzione_a_git/git_commit.png}
        \end{subfigure}
        \hfill
        \begin{subfigure}{0.49\textwidth}
          \centering
          \includegraphics[height=4cm]{introduzione_a_git/git_commit_staging.png}
        \end{subfigure}
      \end{figure}
    \end{minipage}

    \subsubsection{Branch}
    Un branch è una linea di sviluppo, composta da un insieme ordinato di commit collegati in un DAG, il quale inizia dal primo commit del repository e punta all'ultimo commit.\\
    Grazie ai branch è possibile \textbf{lavorare parallelamente} a più versioni del progetto.
    \begin{figure}[H]
      \centering
      \includegraphics[width=0.7\textwidth]{introduzione_a_git/branch.png}
    \end{figure}

    \subsubsection{HEAD}
    L'HEAD è un puntatore alla posizione attuale rispetto alla storia del repository e può essere aggiornato tramite il comando \textit{checkout}.\\
    Solitamente l'HEAD punta ad un branch o ad un tag, qualora invece puntasse ad un commit si parlerebbe di \textbf{Detached HEAD}. Quando ci si trova in questo stato i commit fatti non vengono inseriti in alcun branch, rischiando quindi di andare persi.
    \begin{figure}[H]
      \centering
      \includegraphics[width=0.7\textwidth]{introduzione_a_git/branch_head.png}
    \end{figure}

    \subsubsection{Tag}
    Un tag è un'etichetta per un commit e viene solitamente usato per segnare versioni importandi di un progetto (e.g. \textit{v1.0.0}, \textit{release-2025-09}).

    \subsubsection{Merge}
    Il merge è un'operazione che fonde i cambiamenti tra due branch, facendo in modo che la destinazione contenga entrambi i cambiamenti mentre l'origine rimanga immutata.\\
    Può avvenire mediante tre strategie:
    \begin{itemize}
      \item \textbf{Fast forward:} quando il branch di destinazione non ha commit successivi rispetto a quello che si vuole unire e il sistema sposta semplicemente il puntatore in avanti;
      \item \textbf{Merge commit:} quando i branch hanno sviluppi indipendenti e il sistema crea un nuovo commit con due genitori;
      \item \textbf{Rebase:} il sistema ricrea ogni commit non in comune tra i due branch, mantendo così una storia lineare.
    \end{itemize}
  
    \subsection{Comandi Git}
    Per interfacciarsi con Git vengono messi a disposizione dal sistema diversi comandi:
    \begin{itemize}
      \item \texttt{git init:} inizializza un repository creando una subdirectory .git all'interno della directory corrente; 
      \item \texttt{git status:} mostra lo stato attuale del repository (file tracciati, file modificati, file nello staging, file non tracciati);
      \item \texttt{git diff:} mostra le differenze tra working directory, staging e commit;
      \item \texttt{git add <file>:} aggiunge un file alla staging area (\texttt{git add .} per aggiungere tutti i file modificati);
      \item \texttt{git commit -m "Messaggio":} crea un commit, registrando le modifiche aggiunte con \texttt{git add} nella cronologia del repository;
      \item \texttt{git log:} mostra la lista dei commit effettuati;
      \item \texttt{git branch <nome>:} crea un nuovo branch con il nome indicato, ma \textbf{non ci si sposta};
      \item \texttt{git checkout <nome>:} passa ad un branch esistente spostando l'HEAD;
      \item \texttt{git checkout -b <nome>:} crea un nuovo branch con il nome indicato, per poi \textbf{spostarsi} su quest'ultimo (\texttt{git checkout -b <nome>} = \texttt{git branch <nome>} + \texttt{git checkout <nome>});
      \item \texttt{git fetch:} scarica gli aggiornamenti (commit, branch) dal repository remoto, \textbf{senza merge} col tuo branch;
      \item \texttt{git merge <branch>:} unisce la cronologia del branch in cui ci si trova con quella del branch specificato;
      \item \texttt{git pull:} scarica gli aggiornamenti (commit, branch) dal repository remoto, \textbf{facendo merge} col tuo branch (\texttt{git pull} = \texttt{git fetch} + \texttt{git merge});
      \item \texttt{git push:} invia i commit locali al repository remoto, aggiornando il branch remoto corrispondente; 
    \end{itemize}

    \subsection{Git flow}
\end{document}

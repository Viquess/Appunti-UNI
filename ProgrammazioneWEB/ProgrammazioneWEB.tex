\documentclass[12pt]{article}

\usepackage{graphicx}
\usepackage[italian]{babel}
\usepackage[a4paper, margin=75pt]{geometry}
\usepackage{titlesec}
\usepackage{xcolor}
\usepackage{enumitem}

\graphicspath{{images/}}
\setlist{itemsep=2pt}

\titleformat{\section}
  [block]
  {\raggedleft\LARGE\bfseries}
  {\textcolor{gray}{\scalebox{5}{\thesection}}}
  {0pt}
  {\\[3pt]}

\titlespacing*{\section}
  {0pt}   % rientro sinistro
  {0pt}   % spazio prima della sezione
  {30pt}  % spazio dopo la sezione

\begin{document}
    % --- Copertina ---
    \begin{titlepage}
        \centering

        % --- Logo Sapienza ---
        \includegraphics[width=0.95\textwidth]{logo_sapienza.png}
        
        \vspace*{\stretch{0.2}}
        
        % --- Dati Sapienza ---
        {\LARGE "Sapienza" Università di Roma}\\[3pt]
        {\Large Ingegneria dell'Informazione, Informatica e Statistica}\\[3pt]
        {\large Dipartimento di Informatica}\\[3pt]
        
        % --- Spazio vuoto ---
        \vspace*{\stretch{1}}
        
        % --- Titolo ---
        \hrulefill\\
        \vspace{15pt}
        \textbf{\huge Programmazione WEB}\\
        \vspace{7pt}
        \hrulefill\\
        
        % --- Spazio vuoto ---
        \vspace*{\stretch{2}}
        
        % --- Autore ---
        \textit{\Large Autore}\\[3pt]
        {\Large Vincenzo Bova}\\
        
        % --- Spazio vuoto ---
        \vspace*{\stretch{1}}

        % --- Data ---
        {\large A.A. 2025/2026}\\
    \end{titlepage}

    % --- Indice ---
    \newpage
    \tableofcontents

    % --- Introduzione a Git ---
    \newpage
    \section{Introduzione a Git}
    Durante lo sviluppo di un progetto c'è spesso la necessità di effettuare revisioni, correzioni o modifiche ai file che lo compongono.
    \begin{figure}[h]
        \centering
        \includegraphics[width=0.5\textwidth]{introduzione_a_git/no_git_example.png}
        \caption{Esempio tramite la creazione di nuovi file}
        \label{fig:no_git_example}
    \end{figure}\\
    Gestire ciò utilizzando diversi file, tuttavia, comporta evidenti problemi:
    \begin{itemize}
      \item \textbf{Duplicazione del contenuto:} che rende il sistema inefficiente e aumenta la difficoltà nel mantenere integrità;
      \item \textbf{Assenza di Naming Convention:} che rende impossibile risalire ad uno storico delle modifiche;
      \item \textbf{Autori incerti};
      \item ...
    \end{itemize}
    Per ovviare a ciò sono stati creati i \textbf{sistemi di versionamento}.
\end{document}
